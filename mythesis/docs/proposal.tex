%%
% 开题报告
% modifier: 黄俊杰(huangjj27, 349373001dc@gmail.com)
% update date: 2017-05-14

% 选题目的
\objective{
    \qquad 随着医疗技术的进步,各种新型的医学影像设备已广泛应用在临床诊断中,其中包括计算机断层扫描(CT)、磁共振成像(MRI)、超声成像(UI)等。医学图像中含有非常有用的信息,医生可利用CT及其他医学图像来诊断患者病情,医学图像已逐渐成为临床诊断的主要依据,因此,对医学图像处理的研究具有重要意义。其中,医学图像分割是该领域的研究热点,属于语义分割的一种,图像分割将图像划分成多个解剖学意义的区域,并在此基础上可以计算相应区域的相关定量指标用于辅助临床诊断。
    
    \qquad 在临床诊断中,多模态的医学图像常被用于辅助诊断,而医学图像的标注成本比较昂贵,专家标注一张完整的分割图像大约需要8小时。同时,在医学图像领域,由于各种成像模式具有不同的物理原理,异构域位移的情况更加常见和严重,如不同医院的医学图像的分布差异较大的多中心问题,不同成像原理的跨模态问题等。因此,我们希望利用已标注好的单模态图像,对跨模态的无标注图像进行医学图像分割。
    
}

% 思路
\methodology{
    \qquad 解决这个问题的一种不失为有效的方法是基于学习的方法对一种模态的已有标注数据进行建模并用于另一种模态的图像分割。然而,机器学习方法通常假设训练集和测试集同属于一个数据分布,这个假设在该实际的临床场景中通常不成立,在训练集上训练好的模型在测试集上不能得到很好的泛化。先前的研究表明,测试误差会随着训练集和测试集的分布差异而增加,这就是域位移问题,训练集(源域)和测试集(目标域)的分布具有一定的差异。在不需要目标域的带标注数据的情况下,无监督领域自适应是解决由域位移带来的性能下降的一种方法。
}

% 研究方法/程序/步骤
\researchProcedure{
    \qquad 提出一个新的无监督领域自适应框架用于跨模态医学图像分割。同时考虑到不同模态的医学图像具有相似的解剖学特征,采用解耦表示学习来提取出不同模态的心脏图像的模态特征以及解剖学特征,结合图像自适应和特征自适应,实现跨模态的图像转换以及医学图像分割。
}

% 相关支持条件
\supportment{
    \qquad 该课题是近几年医学图像领域的热门方向,国内外进行了大量的相关研究。此外,Multi-Modality Whole Heart Segmentation Challenge 2017提供了该方向的一个基准数据集,在其上的评价指标能验证方法的有效性。


}

% 进度安排
\schedule{
    \begin{center}
        \begin{tabular}{|l|l|}
            \hline 时间段&计划任务\\
            \hline 1月31日&提交开题报告\\
                    2月15日之前&广泛阅读文献\\
                    2月15日至3月15日&初步完成实验\\
                    3月1日至3月20日&完成初稿\\
                    4月1日至4月中旬&修改论文并定稿\\
            \hline
        \end{tabular}
    \end{center}

}

% 指导老师意见
\proposalInstructions{

}


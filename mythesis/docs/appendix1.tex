\chapter{相关损失函数公式}

\section{交叉熵公式}
交叉熵用于衡量两个概率分布之间的差异,常用于分类问题,语义分割是像素级的一类分类问题。二元交叉熵公式定义为:
\begin{align}
    \mathcal{L}_{C E}(y, \hat{y})=-(y \log (\hat{y})+(1-y) \log (1-\hat{y})).
\end{align}
其中$\hat{y}$为预测值,$y$为真实值。
而在语义分割中,不同语义的区域往往面积(体积)不一致,存在数据不平衡的问题,故常常使用平衡交叉熵(Balanced Cross-Entropy):
\begin{align}
    \mathcal{H}(y, \hat{y})=-(\beta * y \log (\hat{y})+(1-\beta) *(1-y) \log (1-\hat{y})).
\end{align}
其中,$\beta=1-\frac{y}{H*W}$,$H$和$W$为图像的宽度和高度。
\section{Dice损失}
给定两个集合$A$和$B$,二者的Dice分数定义为:
\begin{align}
    \operatorname{Dice}(A,B)=\frac{2|A\cap B|}{|A| + |B|}.
\end{align}
其范围为$\left[0, 1\right]$,值越大表示两个集合重合度越高。

\section{平均表面距离}
对于3D物体的每个体素,如果在其18邻域范围内至少有一个像素不是物体,则认为该体素属于表面,依次遍历所有的体素便得到物体的表面。记两个3维图像的表面为$X$和$Y$,平均表面距离(Average Surface Distance)定义为:
\begin{align}
    \operatorname{ASD}(X,Y) = \frac{\sum_{x\in X} \min_{y\in Y}d(x,y)}{|X|}.
\end{align}
其中$d(\cdot,\cdot)$代表欧氏距离。


\endinput

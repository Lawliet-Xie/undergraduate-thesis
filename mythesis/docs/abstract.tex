%%
% 摘要信息
% 本文档中前缀"c-"代表中文版字段, 前缀"e-"代表英文版字段
% 摘要内容应概括地反映出本论文的主要内容,主要说明本论文的研究目的、内容、方法、成果和结论。要突出本论文的创造性成果或新见解,不要与引言相 混淆。语言力求精练、准确,以 300—500 字为宜。
% 在摘要的下方另起一行,注明本文的关键词(3—5 个)。关键词是供检索用的主题词条,应采用能覆盖论文主要内容的通用技术词条(参照相应的技术术语 标准)。按词条的外延层次排列(外延大的排在前面)。摘要与关键词应在同一页。
% modifier: 黄俊杰(huangjj27, 349373001dc@gmail.com)
% update date: 2017-04-15
%%

\cabstract{

    %无监督领域自适应在医学图像领域中越来越受到关注,旨在解决医学图像中常见的域位移现象导致的模型性能下降问题。这篇文章提出了一个新的无监督域自适应框架用于跨模态的图像分割。该框架基于解耦表示学习来提取医学图像中的解剖学特征和模态特征,同时利用该解剖学特征来完成图像分割任务。我们使用了循环一致损失和对抗学习来进行图像层面上的领域自适应,实现跨模态之间的图像转换。我们在不成对的多模态CT-MRI心脏数据集上验证了我们的方法,实验结果表明我们框架改进域位移问题的有效性。
    无监督领域自适应原理在医学图像领域中受到越来越多的关注,旨在解决医学图像中常见的域位移现象导致的分割模型性能下降问题。本文提出了一个新的无监督域自适应框架用于跨模态的医学图像分割。该框架基于解耦表示学习来提取医学图像中的解剖学特征和模态特征,进而利用解剖学特征来完成分割任务。同时使用了循环一致损失和对抗学习来进行图像层面上的领域自适应,实现跨模态的图像转换。该方法在不成对的多模态CT-MRI心脏数据集上的实验结果表明此框架改进域位移问题的有效性并具有良好的图像分割性能。
}
% 中文关键词(每个关键词之间用“,”分开,最后一个关键词不打标点符号。)
\ckeywords{无监督领域自适应,医学图像分割,解耦表示学习,图像转换,对抗学习}

\eabstract{
    % 英文摘要及关键词内容应与中文摘要及关键词内容相同。中英文摘要及其关键词各置一页内。
    Unsupervised domain adaptation has become an important and hot topic in recent studies on medical image computing, aiming to recover performance degration when applying the neural networks to new testing domains. This paper proposes a novel unsupervised domain adaptation framework to effectively tackle the severe problem of domain shift.  Our proposed method extracts anatomy features and modality features via disentangled representation learning and enhances the anatomy features towards the segmentation task. To handle unpaired training data, we fuse the cross-cycle consistency loss and adversarial learning to achieve image-to-image translation between different domains. We have extensively validated our method with a challenging application of cross-modality medical image segmentation of cardiac structures. Experimental results demonstrate the effectiveness of our method.
}
% 英文文关键词(每个关键词之间用,分开, 最后一个关键词不打标点符号。)
\ekeywords{Unsupervised domain adaptation, medical image segmentation, disentangled representation learning, image-to-image translation, adversarial learning}

